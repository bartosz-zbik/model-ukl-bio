\documentclass[12pt]{article}
\usepackage[usenames]{color} %used for font color
\usepackage{amssymb} %maths
\usepackage{amsmath} %maths
\usepackage[utf8]{inputenc} %useful to type directly diacritic characters
\usepackage[T1]{fontenc}
\usepackage{polski}
\usepackage{pgffor}

% \usepackage{listings}
\usepackage[theme=default-plain]{jlcode} % https://github.com/wg030/jlcode
\usepackage{graphicx}  %grafika

%\usepackage{subcaption} %dwa rysunki obok siebie
%\graphicspath{ {./rysunki/} } %skąd ma pobierać grafikę
%\usepackage{svg} % for graphics in svg, with the first compilation enable: "-shell-escape" to use tools to convert svg
%\usepackage{csvsimple} % to insert csv files 

\usepackage[a4paper, left=3cm, right=3cm, top=1.5cm, bottom=2cm]{geometry}


\setlength{\parskip}{10pt}
\setlength{\parindent}{0pt}

\title{Model gradacji owadów}
\author{Bartosz Zbik}
\date{2024-05-18} %format jest dowolny(może być nawet miesiąc słownie

\begin{document}
\maketitle %tworzy tytuł dokumentu

Wyjściowe równanie
\begin{equation}
\frac{dN}{dt} = r_0 N  \left ( 1 - \frac{N}{K} \right ) - \frac{BN^2}{A^2 + N^2}
\end{equation}
możemy zredukować (przy pomocy linowych transformacji) do
\begin{equation}
\frac{dn}{d\tau} = r n  \left ( 1 - \frac{n}{k} \right ) - \frac{n^2}{1 + n^2},
\end{equation}
gdzie
\begin{equation}
n = \frac{N}{A}, ~~ \tau = \frac{B}{A} t, ~~ r = \frac{A}{B} r_0, ~~ k = \frac{1}{A} k.
\end{equation}

\section{Kod}
\jlinputlisting{solution.jl}
\clearpage


\section{Wizualizacja wyników}
We wszystkich przypadkach układ dążył do stanu stacjonarnego.
 
\subsection{Rozwiązania dla $k=8$}
Dla $k=8$ można zauważyć, że są co najmniej dwa stabilne stany stacjonarne.
Jeden dla $n^*$ gdzieś pomiędzy $0.6$, a $0.7$.
Drugi  dla $n^*$ gdzieś pomiędzy $4.5$, a $5.0$.
\foreach \n in {1,...,4}{
\includegraphics[width=\linewidth]{out/model_k8_\n}

}
\clearpage


\subsection{Rozwiązania dla $k=5$}
\foreach \n in {1,...,2}{
\includegraphics[width=\linewidth]{out/model_k5_\n}

}





\end{document}