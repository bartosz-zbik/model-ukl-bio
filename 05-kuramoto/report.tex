\documentclass[12pt]{article}
\usepackage[usenames]{color} %used for font color
\usepackage{amssymb} %maths
\usepackage{amsmath} %maths
\usepackage[utf8]{inputenc} %useful to type directly diacritic characters
\usepackage[T1]{fontenc}
\usepackage{polski}
\usepackage{pgffor}


% \usepackage{listings}
\usepackage[theme=default-plain]{jlcode} % https://github.com/wg030/jlcode
\usepackage{graphicx}  %grafika

%\usepackage{subcaption} %dwa rysunki obok siebie
%\graphicspath{ {./rysunki/} } %skąd ma pobierać grafikę
%\usepackage{svg} % for graphics in svg, with the first compilation enable: "-shell-escape" to use tools to convert svg
%\usepackage{csvsimple} % to insert csv files 

\usepackage[a4paper, left=3cm, right=3cm, top=1.5cm, bottom=2cm]{geometry}


\setlength{\parskip}{10pt}
\setlength{\parindent}{0pt}

\title{Model Kuramoto}
\author{Bartosz Zbik}
\date{2024-05-11} %format jest dowolny(może być nawet miesiąc słownie

\begin{document}
\maketitle %tworzy tytuł dokumentu

Model dany jest przez układ $N$ równań
\begin{equation}
\dot \theta_i = \omega_i + \frac{K}{N} \sum_{j=1}^N \sin (\theta_j - \theta_i)
\end{equation}

Definiujemy parametr prządku $r$ i fazę $\psi$
\begin{equation}
r e^{i\psi} = \frac{1}{N} \sum_{j=1}^N e^{i\theta_j},
\end{equation}
co pozwala przepisać układ równań w postaci
\begin{equation}
\label{eq:nice-model}
\dot \theta_i = \omega_i - r K \sin (\theta_i - \psi)
\end{equation}

Rozwiązywałem go obliczając w każdym kroku całkowania najpierw $r$ i $\psi$, a potem poszczególne przyrosty wynikające z równania~(\ref{eq:nice-model}) co sprowadzało się do złożoności $\mathcal{O}(3N)$ (albo $\mathcal{O}(2N)$ w zależności, czy $e^{i\theta}$ wymaga obliczenia dwóch funkcji trygonometrycznych czy jednej) , a nie $\mathcal{O}(N^2)$.

Funkcje \texttt{kuramoto!} i \texttt{kuramotoorder} są wykorzystywane przez solver do rozwiązania równania.
Warunki początkowe $\theta_i(0)$ oraz parametry $\omega_i$ są losowane, a cała procedura schowana jest w funkcjach \texttt{findsol} (dla $\omega \sim \mathcal{N}(0,1)$) oraz \texttt{findcauchysol} (dla $\omega$ z rozkładu Cauchy'ego).
Funkcje \texttt{asymptoticnormal} i \texttt{asymptoticcauchy} służą już tylko do wygenerowania rysunków i nie ma tam żadnej logiki związanej z samym rozwiązaniem problemu.


\section{Kod}
\jlinputlisting{solution.jl}
\clearpage


\section{Wizualizacja wyników}
Przy rozwiązywaniu przyjąłem $N=512$ i dla każdej wartości $K$ wygenerowałem 30 rozwiązań. 
W rozwiązaniu jako estymator parametru porządku $r$ brałem średnią po realizacjach, co nie zawsze dawało wynik zgodny z teorią. 
Prawdopodobnie dlatego, że $r \in [0,1]$, więc przykładowo dla $K < K_c$ fluktuacje $r$ jedynie zawyżają $<r>$.

Na wszystkich poniższych rysunkach szare krzywe odpowiadają poszczególnym trajektoriom, ciągła niebieska linia odpowiada średniemu $r$ liczonemu po trajektoriach, a linia przerywana odpowiada analitycznie wyznaczonej wartości $r$.
W nagłówkach rysunków podawałem $r$ i $<r>$ z dokładnością do trzech cyfr po przecinku. 

\subsection{Rozkład Normalny}
Dla $\omega$ pochodzącego ze standardowego rozkładu normalnego
\begin{equation}
g(\omega) = \frac{1}{\sqrt{2\pi}} e^{\frac{-\omega^2}{2}}
\end{equation}
Wartość krytyczną wyznaczamy dzieląc równianie
\begin{equation}
r = K r \int_{-\pi/2}^{\pi/2} \cos^2{(\theta)} g(Kr \sin \theta) d \theta 
\end{equation}
przez $r$ i przechodząc do granicy $r \to 0^+$. Dostajemy
\begin{equation}
1 = K \int_{-\pi/2}^{\pi/2} \cos^2{(\theta)} \frac{1}{\sqrt{2\pi}} d \theta = \frac{\pi K}{2\sqrt{2\pi}}.
\end{equation}
Zatem
\begin{equation}
K_c = \frac{2\sqrt{2}}{\sqrt{\pi}}
\end{equation}


\foreach \n in {1,...,9}{
\begin{figure}[h]
\centering
\includegraphics[width=\linewidth]{out/kuramoto_normal_\n}
\end{figure}
}
\clearpage


\subsection{Rozkład Cauchy'ego}
Dla $\omega$ pochodzącego z rozkładu Cauchy'ego
\begin{equation}
g(\omega) = \frac{1}{\pi} \frac{1}{1 + \omega^2}.
\end{equation}
Wyznaczamy parametr krytyczny z równania (całkowanie w Matematice)
\begin{equation}
r = K r \int_{-\pi/2}^{\pi/2} \cos^2{(\theta)} g(Kr \sin \theta) d \theta =
2 K r \frac{1}{2 + 2\sqrt{1+K^2r^2}},
\end{equation}
co po przekształceniu daje
\begin{equation}
r = \sqrt{\frac{K-2}{K}} \quad \text{dla}~K>2
\end{equation}
oraz wartość krytyczną
\begin{equation}
K_c = 2
\end{equation}


\foreach \n in {1,...,9}{
\begin{figure}[h]
\centering
\includegraphics[width=\linewidth]{out/kuramoto_cauchy_\n}
\end{figure}
}


\end{document}